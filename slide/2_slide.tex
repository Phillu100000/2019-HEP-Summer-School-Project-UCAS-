\documentclass{beamer}
\usetheme{Berlin}
\usepackage[UTF8,scheme=plain]{ctex}
\usepackage{amsmath}
\usepackage{amsfonts}
\usepackage{amssymb}
\usepackage{mathrsfs}
\usepackage{bm}
\usepackage{graphicx}
\usepackage{float}
\usepackage{subfigure}
\usepackage{natbib}
\newcommand{\dif}{\mathop{}\!\mathrm{d}}

\title{Measurement of the absolute branching fraction of the inclusive semileptonic $\Lambda_c^+$ decay}
\author{严启宇,黄吉鸿,卢玫澍}
\institute{UCAS}
\date{2019/7/25}

\begin{document}
\bibliographystyle{apalike}
\begin{frame}
    \titlepage
\end{frame}

\begin{frame}
    \frametitle{Outline}
    \tableofcontents
\end{frame}

\section{物理背景}
\subsection{关于$\Lambda_c^+$}
\begin{frame}
    \subsectionpage
\end{frame}

\begin{frame}
    $\Lambda_c^+$重子是含有charm夸克的重子中最轻的, 这一粒子在上世纪70年代年被发现.\cite{knapp1976observation}\cite{abrams1980observation} 其夸克组分为$udc$. $\Lambda_c^+$有众多衰变路径, PDG上给出了现有的测量.

\end{frame}
\begin{frame}
    我们所报告的研究\cite{ablikim2018measurement}的是对于$\Lambda_c^+$的半轻衰变的合并分支比的测量. 此前$\Lambda_c^+ \rightarrow \Lambda e^+ \nu_e$  $(3.63 \pm 0.43)\%$这个末态被此前测定其绝对分支比为$(3.63 \pm 0.43)\%$.


    这与更早的MARK II合作组的测量: $(4.5 \pm 1.7)\%$(对于$\Lambda_c^+ \rightarrow X e^+ \nu_e$而言的合并分支比). 然而这一实验的较大的误差导致不能判断在这背后是否有未被发现的新的衰变模式. 于是, 准确的测定半轻衰变的分支比就是这篇文章的目的.
\end{frame}



\begin{frame}[allowframebreaks]
    \bibliography{ref}
\end{frame}

\end{document}