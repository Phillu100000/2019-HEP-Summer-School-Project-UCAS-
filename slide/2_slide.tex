\documentclass{beamer}
\usetheme{Berlin}
\usepackage[UTF8,scheme=plain]{ctex}
\usepackage{amsmath}
\usepackage{amsfonts}
\usepackage{amssymb}
\usepackage{mathrsfs}
\usepackage{bm}
\usepackage{graphicx}
\usepackage{float}
\usepackage{subfigure}
\usepackage{natbib}
\newcommand{\dif}{\mathop{}\!\mathrm{d}}

\title{Measurement of the absolute branching fraction of the inclusive semileptonic $\Lambda_c^+$ decay}
\author{严启宇,黄吉鸿,卢玫澍}
\institute{UCAS}
\date{2019/7/25}

\begin{document}
\bibliographystyle{apalike}
\begin{frame}
    \titlepage
\end{frame}

\begin{frame}
    \frametitle{Outline}
    \tableofcontents
\end{frame}

\section{物理背景}
\subsection{关于$\Lambda_c^+$}
\begin{frame}
    \subsectionpage
\end{frame}

\begin{frame}
    $\Lambda_c^+$重子是含有charm夸克的重子中最轻的, 这一粒子在上世纪70年代年被发现.\cite{knapp1976observation}\cite{abrams1980observation} 其夸克组分为$udc$. $\Lambda_c^+$有众多衰变路径, PDG上给出了现有的测量

    我们所报告的研究\cite{ablikim2018measurement}背景是来自于
\end{frame}


\begin{frame}[allowframebreaks]
    \bibliography{ref}
\end{frame}

\end{document}